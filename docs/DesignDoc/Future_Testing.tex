\newpage
\section{Manufactured Product Testing}

\subsection*{Common Scenarios (Tests 1–10)}

\subsection*{1. Valid PIN Entry Test}
\subparagraph{Test Goals and Purpose}
\begin{itemize}
    \item Verify that a recognized 4-digit PIN unlocks the door without unnecessary delay.
    \item Check that the system logs this event as a successful access in Firestore.
\end{itemize}
\subparagraph{How We Test It}
\begin{itemize}
    \item Select a PIN known to be valid from the internal list.
    \item Enter it once and observe the system's response, both physical and digital.
    \item Look for updates to the LOCK\_STATE value in Firestore.
\end{itemize}
\subparagraph{Expectations of Test}
\begin{itemize}
    \item The bolt should retract within roughly half a second (this might vary slightly).
    \item LOCK\_STATE is set to 1 in Firestore upon unlock.
    \item There are no anomalies or unexpected entries in the logs.
\end{itemize}

\subsection*{2. Invalid PIN Lockout Test}
\subparagraph{Test Goals and Purpose}
\begin{itemize}
    \item Ensure the system prevents further PIN attempts after multiple consecutive failures.
    \item Confirm some form of lockout indication is presented, such as a light or alarm.
\end{itemize}
\subparagraph{How We Test It}
\begin{itemize}
    \item Enter three clearly incorrect PINs in under 30 seconds.
    \item Check the LOCKOUT\_COUNT value and observe for a visible or audible signal.
\end{itemize}
\subparagraph{Expectations of Test}
\begin{itemize}
    \item PIN entries are blocked for about 60 seconds following the third failure.
    \item Lockout state is visually or audibly indicated (could be LED or buzzer).
    \item Firestore logs each incorrect attempt, along with the lockout status.
\end{itemize}

\subsection*{3. Emergency PIN Test}
\subparagraph{Test Goals and Purpose}
\begin{itemize}
    \item Test whether a designated emergency PIN can override a lockout state.
    \item Ensure that using the emergency PIN resets the system's lockout logic.
\end{itemize}
\subparagraph{How We Test It}
\begin{itemize}
    \item Trigger a lockout by intentionally failing three PIN entries.
    \item Once lockout is active, input the emergency PIN.
    \item Monitor Firestore to verify an EMERGENCY\_UNLOCK entry is created.
\end{itemize}
\subparagraph{Expectations of Test}
\begin{itemize}
    \item The door should unlock immediately, without hesitation.
    \item Firestore logs the use of the emergency PIN distinctly.
    \item The system clears any active lockout status to resume normal operation.
\end{itemize}

\subsection*{4. OTP Entry Test}
\subparagraph{Test Goals and Purpose}
\begin{itemize}
    \item Confirm that an app-generated one-time password (OTP) grants access as expected.
    \item Ensure the OTP mechanism is time-bound and secure against reuse.
\end{itemize}
\subparagraph{How We Test It}
\begin{itemize}
    \item Request a 5-minute valid OTP from the app.
    \item Enter the OTP using the keypad.
    \item Monitor the system and database for changes.
\end{itemize}
\subparagraph{Expectations of Test}
\begin{itemize}
    \item Firestore captures the OTP with a clear expiry time.
    \item Valid OTPs allow immediate unlock.
    \item Expired or incorrect OTPs are properly rejected with no unlock attempt.
\end{itemize}

\subsection*{5. OTP Expiry Test}
\subparagraph{Test Goals and Purpose}
\begin{itemize}
    \item Ensure expired OTPs don’t grant access, even if they were once valid.
    \item Confirm that users are informed clearly when an OTP is no longer usable.
\end{itemize}
\subparagraph{How We Test It}
\begin{itemize}
    \item Generate an OTP and wait for the full 5-minute window to pass.
    \item Attempt to use the expired code via the keypad.
\end{itemize}
\subparagraph{Expectations of Test}
\begin{itemize}
    \item Lock remains unchanged and secure—no movement or unlock occurs.
    \item Firestore logs the expired OTP attempt explicitly.
    \item App and/or interface display an "OTP expired" notification.
\end{itemize}

\subsection*{6. Rapid Wrong PIN Alarm Test}
\subparagraph{Test Goals and Purpose}
\begin{itemize}
    \item Ensure multiple fast wrong PIN entries trigger audible alarm.
    \item Verify lock does not unlock after wrong entries.
    \item Check alarm stops after timeout or correct PIN.
\end{itemize}
\subparagraph{How We Test It}
\begin{itemize}
    \item Enter 5 wrong PINs within 10s.
    \item Observe alarm sound and LOCK\_STATE remains 0.
    \item Attempt correct PIN after alarm to confirm reset.
\end{itemize}
\subparagraph{Expectations of Test}
\begin{itemize}
    \item Alarm beeps continuously for spec duration (e.g., 30s).
    \item Bolt remains locked (LOCK\_STATE = 0).
    \item Firestore logs each failed attempt and alarm event.
    \item Correct PIN after alarm silences alarm and unlocks door.
\end{itemize}

\subsection*{7. Concurrent Unlock Command Test}
\subparagraph{Test Goals and Purpose}
\begin{itemize}
    \item Verify two family phones sending unlock at once doesn't confuse the system.
    \item Ensure only one actuation happens.
    \item Confirm both requests are logged.
\end{itemize}
\subparagraph{How We Test It}
\begin{itemize}
    \item From two phones, send unlock API calls within 100 ms.
    \item Watch bolt movement and LOCK\_STATE.
    \item Check Firestore for two event entries.
\end{itemize}
\subparagraph{Expectations of Test}
\begin{itemize}
    \item Bolt retracts only once.
    \item LOCK\_STATE = 1 after first command.
    \item Both commands logged with timestamps.
    \item No errors or retries triggered.
\end{itemize}

\subsection*{8. Offline Fallback PIN Test}
\subparagraph{Test Goals and Purpose}
\begin{itemize}
    \item Ensure local PIN entry still works if Wi-Fi drops.
    \item Verify lock uses cached PIN list.
    \item Confirm Firestore syncs later.
\end{itemize}
\subparagraph{How We Test It}
\begin{itemize}
    \item Disable Wi-Fi on ESP32, enter valid PIN.
    \item Observe bolt movement and local log.
    \item Re-enable Wi-Fi and check Firestore sync.
\end{itemize}
\subparagraph{Expectations of Test}
\begin{itemize}
    \item Door unlocks locally (LOCK\_STATE = 1 locally).
    \item Event cached and later pushed to Firestore.
    \item No user-perceived delay in unlocking.
    \item Sync event marked with “offline” flag.
\end{itemize}

\subsection*{9. Low-Battery Notification Test}
\subparagraph{Test Goals and Purpose}
\begin{itemize}
    \item Check notification when battery falls below 10 %.
    \item Ensure unlock still works at low battery.
    \item Verify UI warning persists until charge.
\end{itemize}
\subparagraph{How We Test It}
\begin{itemize}
    \item Drain battery to 9 %.
    \item Press unlock PIN and observe behavior.
    \item Check mobile app for low-battery alert.
\end{itemize}
\subparagraph{Expectations of Test}
\begin{itemize}
    \item Door still unlocks (LOCK\_STATE = 1).
    \item App shows persistent “Low Battery” message.
    \item Firestore logs battery level event.
    \item Warning clears only when battery > 20 %.
\end{itemize}

\subsection*{10. Battery Fully Drained Test}
\subparagraph{Test Goals and Purpose}
\begin{itemize}
    \item Verify lock fails to actuate when battery is dead.
    \item Ensure system logs failure and alerts user.
    \item Confirm physical key still works.
\end{itemize}
\subparagraph{How We Test It}
\begin{itemize}
    \item Let battery drop to 0 %.
    \item Enter valid PIN and observe no bolt movement.
    \item Try physical key override.
\end{itemize}
\subparagraph{Expectations of Test}
\begin{itemize}
    \item Bolt does not move on electronic command.
    \item Firestore logs “Battery Depleted” error.
    \item Physical key unlocks door.
    \item App advises “Replace Battery” notification.
\end{itemize}






















\newpage
\subsection*{Less Common Scenarios (Tests 31–40)}

\subsection*{31. Multi-factor Auth Proximity Test}
\subparagraph{Test Goals and Purpose}
\begin{itemize}
    \item Confirm the system requires both a valid PIN and Bluetooth proximity to unlock.
    \item Observe behavior if Bluetooth disconnects mid-process.
\end{itemize}
\subparagraph{How We Test It}
\begin{itemize}
    \item Start PIN entry and slowly move the paired phone out of Bluetooth range.
    \item Review logs to see how the system treats incomplete multi-factor input.
\end{itemize}
\subparagraph{Expectations of Test}
\begin{itemize}
    \item Door only unlocks when both factors are simultaneously validated.
    \item Firestore logs clearly show whether it was the PIN or BLE that failed.
\end{itemize}

\subsection*{32. Time-zone Mismatch Test}
\subparagraph{Test Goals and Purpose}
\begin{itemize}
    \item Ensure that timing-sensitive operations work properly despite device time mismatches.
    \item Verify that scheduled actions and OTPs remain aligned regardless of local time zones.
\end{itemize}
\subparagraph{How We Test It}
\begin{itemize}
    \item Set the phone to Pacific Time and the lock to Coordinated Universal Time.
    \item Generate and use an OTP, and observe if the mismatch causes issues.
\end{itemize}
\subparagraph{Expectations of Test}
\begin{itemize}
    \item OTPs still work during their intended validity window.
    \item Scheduled events (e.g., auto-lock) trigger based on synchronized absolute times.
\end{itemize}

\subsection*{33. DST Transition Auto-lock Test}
\subparagraph{Test Goals and Purpose}
\begin{itemize}
    \item Confirm that daylight saving time transitions don't disrupt automated locking.
\end{itemize}
\subparagraph{How We Test It}
\begin{itemize}
    \item Schedule an auto-lock event for 2:00 AM on the day of the DST shift.
    \item Observe actual lock behavior before, during, and after the transition.
\end{itemize}
\subparagraph{Expectations of Test}
\begin{itemize}
    \item Only one auto-lock event occurs, even with the clock shift.
    \item No skipped or duplicated scheduling happens.
\end{itemize}

\subsection*{34. Admin vs. Guest Role Change Test}
\subparagraph{Test Goals and Purpose}
\begin{itemize}
    \item Validate that role-based access control takes effect immediately after role changes.
\end{itemize}
\subparagraph{How We Test It}
\begin{itemize}
    \item Change a user's role (e.g., from guest to admin) directly in Firestore.
    \item Attempt access under both roles shortly after the change.
\end{itemize}
\subparagraph{Expectations of Test}
\begin{itemize}
    \item The new role is honored without requiring system restart or delay.
    \item Access permissions align with the updated role instantly.
\end{itemize}

\subsection*{36. RF Interference Packet Loss Test}
\subparagraph{Test Goals and Purpose}
\begin{itemize}
    \item Simulate garage opener RF noise to test packet loss.
    \item Verify retry logic for cloud commands.
    \item Confirm no false unlocks.
\end{itemize}
\subparagraph{How We Test It}
\begin{itemize}
    \item Broadcast 433 MHz noise near ESP32 antenna.
    \item Send unlock API calls and count retries.
    \item Observe LOCK\_STATE and error logs.
\end{itemize}
\subparagraph{Expectations of Test}
\begin{itemize}
    \item Commands retried per spec (e.g., 3 times).
    \item Bolt only moves on successful retry.
    \item Firestore logs transient network errors.
    \item No unintended lock/unlock.
\end{itemize}

\subsection*{37. Magnetic Interference Test}
\subparagraph{Test Goals and Purpose}
\begin{itemize}
    \item Place strong magnet near hall sensor to test false state.
    \item Ensure system ignores brief sensor spikes.
    \item Verify correct state after magnet removal.
\end{itemize}
\subparagraph{How We Test It}
\begin{itemize}
    \item Hold magnet 1 cm from sensor for 30 s.
    \item Monitor lock state readings.
    \item Remove magnet and check sensor recovery.
\end{itemize}
\subparagraph{Expectations of Test}
\begin{itemize}
    \item Any false “unlocked” readings ignored if < 2 s.
    \item Bolt stays locked (LOCK\_STATE = 0).
    \item Firestore logs sensor glitch with duration.
    \item Normal operation resumes immediately.
\end{itemize}

\subsection*{38. Corroded Contact Intermittency Test}
\subparagraph{Test Goals and Purpose}
\begin{itemize}
    \item Simulate moisture-induced corrosion on battery terminals.
    \item Verify intermittent power behavior handling.
    \item Ensure safe shutdown on undervoltage.
\end{itemize}
\subparagraph{How We Test It}
\begin{itemize}
    \item Apply salt-water spray to contacts.
    \item Cycle power and observe voltage drops.
    \item Attempt unlock during drop events.
\end{itemize}
\subparagraph{Expectations of Test}
\begin{itemize}
    \item Device resets gracefully on undervoltage.
    \item Bolt remains locked during power dips.
    \item Firestore logs power instability events.
    \item System recovers once voltage stabilizes.
\end{itemize}

\subsection*{39. UV Exposure Housing Test}
\subparagraph{Test Goals and Purpose}
\begin{itemize}
    \item Expose plastic housing to UV light to test warping.
    \item Confirm no interference with mechanical parts.
    \item Verify seal integrity remains.
\end{itemize}
\subparagraph{How We Test It}
\begin{itemize}
    \item Place lock under UV lamp (340 nm) for 48 h.
    \item Inspect housing dimensions and seal fit.
    \item Operate lock/unlock 5 times post-exposure.
\end{itemize}
\subparagraph{Expectations of Test}
\begin{itemize}
    \item No significant dimensional change (> 1 mm).
    \item Seals remain watertight.
    \item Bolt operation unaffected.
    \item Firestore logs pass/fail UV test result.
\end{itemize}

\subsection*{40. Heavy Vibration Hall-Sensor Test}
\subparagraph{Test Goals and Purpose}
\begin{itemize}
    \item Shake lock with heavy-equipment vibration profile.
    \item Ensure hall-sensor readings stay stable.
    \item Confirm no loosening of fasteners.
\end{itemize}
\subparagraph{How We Test It}
\begin{itemize}
    \item Mount on shaker table at 1 g, 20–100 Hz for 15 min.
    \item Monitor sensor output and bolt status.
    \item Inspect hardware after test.
\end{itemize}
\subparagraph{Expectations of Test}
\begin{itemize}
    \item Sensor jitter < 5 of normal signal.
    \item Bolt/unlock still works correctly.
    \item No loose screws or connectors.
    \item Firestore logs vibration test data.
\end{itemize}