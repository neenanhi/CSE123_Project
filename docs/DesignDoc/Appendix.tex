\section{Appendices}

\textcolor{teal}{\subsection{Appendix 1 - Problem Formulation}}

\subsubsection{6-3-5 Method}
\subsubsection*{Hardware \& Mechanics}
\begin{itemize}
    \item use a servo motor for deadbolt control
    \item hall effect sensor to detect if the door is open or closed
    \item implement linear actuator for silent locking
    \item maybe anti-backlash gears to reduce mechanical play
    \item consider designing a modular lock housing with 3D-printed parts
    \item have a backup battery to ensure operation during power outages
\end{itemize}

\subsubsection*{Smartphone Integration}
\begin{itemize}
    \item develop a smartphone app for remote locking/unlocking
    \item push notifications for lock activity (ex: ``Door locked at 3:45 PM'')
    \item have biometric authentication (Face ID or fingerprint)
    \item Use Bluetooth for proximity-based auto-unlock
    \item NFC support for quick unlocking via phone tap
    \item include multi-user access control with time-based permissions
\end{itemize}

\subsubsection*{Security Features}
\begin{itemize}
    \item Implement AES-256 encryption for communication between the lock and phone
    \item maybe two-factor authentication for app access
    \item develop a tamper-detection alarm if the lock is forced
    \item lockdown mode for multiple failed attempts
    \item include a manual override mechanism in case of system failure
    \item utilize rolling codes for Bluetooth pairing to prevent hacking
\end{itemize}

\subsubsection*{Software \& Control}
\begin{itemize}
    \item ESP32 microcontroller for Wi-Fi and Bluetooth control
    \item Develop a closed-loop system to verify if the lock is engaged properly
    \item make a real-time event log accessible via the app
    \item add a scheduling feature for auto-locking at specific times
    \item guest mode with temporary passcodes
    \item Integration with voice assistants like Alexa or Google Assistant
\end{itemize}

\subsubsection*{Advanced Features}
\begin{itemize}
    \item GPS stuff
    \item add geofencing to lock or unlock based on the user's location
    \item integrate with smart home platforms like Home Assistant
    \item use AI-based behavior analysis to suggest locking patterns
    \item add a camera with facial recognition for auto-unlock
    \item enable remote firmware updates
    \item learn database SQL
    \item use solar charging for battery-powered locks
\end{itemize}

% BRAINSTORMING -----------------------------

\subsubsection{Brainstorming}

\subsubsection*{Adam Wu}
\begin{itemize}
    \item As a person who has amnesia, I would like to be able to find my keys anytime so that when I forget where I place them, I can find them.
    \begin{itemize}
        \item Having a "find my" solution with a key.
    \end{itemize}
    \item As a person who loves security, I would like to have the best lock for my house so that lock pickers are not able to pick my lock.
    \begin{itemize}
        \item Making an “authentication” key that resets the key code within a set time, making it harder for hackers to unlock the door.
    \end{itemize}
    \item As a person who is always last-minute out the door, I fear forgetting to lock the door when I close it.
    \begin{itemize}
        \item Auto-locking door when a person closes the door.
    \end{itemize}
    \item As a person who often forgets to bring their keys, I am scared of getting locked out.
    \begin{itemize}
        \item Having a notification from the key to the phone that alerts: “keys are not close by to you.”
    \end{itemize}
    \item As a parent, I am scared of my kids forgetting their keys and locking themselves out of their room.
    \begin{itemize}
        \item Creating a “master key” that only parents/admins can use to unlock specific doors.
    \end{itemize}
    \item Concerned about key battery life.
    \begin{itemize}
        \item Send a notification to the user when the key is on low battery.
    \end{itemize}
\end{itemize}

\subsubsection*{Nathaniel Laurente}
\begin{itemize}
    \item Key has the ability to notify the user when too far away from the user’s phone/body.
    \item Key deactivates/won’t be able to open the door if too far away from the owner.
    \item “Tap to Pay” technology concept.
    \begin{itemize}
        \item Unlocks the door like a credit card tap on a phone.
        \item If too complex, explore alternative ways to unlock the door.
        \item Eliminates the need for a physical key.
        \item Prevents stolen keys from working if the user still has their phone.
    \end{itemize}
    \item Secure deactivation of the key when too far from the user.
    \begin{itemize}
        \item Possible solution: Use the user's phone for deactivation.
    \end{itemize}
    \item One-time password generator between lock and key to ensure only this exact key can enter the house.
    \item Backup way to get into the house if the user forgets/loses their key.
    \begin{itemize}
        \item Pin access code.
        \item App allows for 2FA authentication using a thumbprint and/or Face ID.
    \end{itemize}
    \item Will the battery last long enough for multiple years?
\end{itemize}

\subsubsection*{Neena Nguyen}
\begin{itemize}
    \item Existing smart lock solutions:
    \begin{itemize}
        \item Smart locks for dorm rooms using mobile apps, passcodes, and scanners.
    \end{itemize}
    \item Who will use this lock?
    \begin{itemize}
        \item People with memory issues (elderly, ADHD).
        \item University students in dorm rooms.
        \item Student ID scanner integration.
        \item Parents with small children (child-proof locks).
    \end{itemize}
    \item Features for parental control.
    \begin{itemize}
        \item Locks after a curfew time.
        \item Prevents children from unlocking without parental approval.
        \item Alerts parents when kids come home from school.
    \end{itemize}
    \item What kind of door lock will it be?
    \begin{itemize}
        \item Facial recognition (requires camera and database knowledge).
        \item Logs entry and exit timestamps.
        \item Digital passcode through an app.
        \item Auto-relocking mechanism after failed attempts.
        \item Bluetooth detection for unlocking within a certain range.
        \item Dual authentication (PIN + scan).
        \item Optional security trigger after specific hours.
        \item Alerts when the door is left unlocked for too long.
        \item Auto-locking after prolonged unlocking.
        \item Detection system to check if the key is on the person.
        \item Prevents intruders from entering without a key.
    \end{itemize}
\end{itemize}

\subsubsection*{Jackson Kennedy}
\begin{itemize}
    \item Normal keys can be lock-picked, but digital keys can be secured based on a communication protocol.
    \item Secure authentication methods.
    \begin{itemize}
        \item PIN authentication with 2FA.
        \item Optimal PIN length (e.g., 4-digit PIN has 1,048,576 combinations).
        \item Brute force prevention strategies.
    \end{itemize}
    \item Preventing communication protocol vulnerabilities.
    \begin{itemize}
        \item What protocol should be used? (Bluetooth has vulnerabilities and short range.)
        \item Cloud-based solutions rely on third-party vendor security.
        \item What information needs to be transferred? (Video data, authentication signals?)
    \end{itemize}
    \item Lock activation logic.
    \begin{itemize}
        \item How exactly will the lock know when to unlock? (Sending a 0 or 1 signal based on specific conditions?)
    \end{itemize}
    \item Security and alerting technologies.
    \begin{itemize}
        \item Sensors to detect nearby people.
        \item Hidden camera or biometric verification for identity confirmation.
    \end{itemize}
    \item Scheduling and timed access.
    \begin{itemize}
        \item Physical locks do not have scheduling options.
        \item Implement timed unlocking (e.g., unlock for 15 minutes for a babysitter).
        \item Extra verification to prevent intruders from exploiting schedules.
    \end{itemize}
\end{itemize}
\newpage

% Begin Morphological Charts ----------------------------------

\subsubsection{Morphological Charts}

\begin{figure}[!ht]
    \centering
    \includegraphics[width=1\linewidth]{./img/p1mm.png}
    \caption{Physical Solution Chart}
    \label{fig:p1mm}
\end{figure}
\newpage
\begin{figure}[!ht]
    \centering
    \includegraphics[width=1\linewidth]{./img/p2mm.png}
    \caption{Interface Solutions}
    \label{fig:p2mm}
\end{figure}
\newpage

% Begin Mind Map ---------------------------------------------

\subsubsection{Mind Map}

\begin{figure}[!ht]
    \centering
    \includegraphics[width=140mm,scale=0.5]{./img/mindmapsmartlock.png}
    \caption{Mind Map}
    \label{fig:mindmapsmartlock}
\end{figure}

\subsubsection{Concept Selection}
\subsubsection*{Weighting Factors}
\begin{tabular}{ll}
    \toprule
    Criteria & Weight (\%) \\
    \midrule
    Security & 40 \\
    Cost & 30 \\
    Power Consumption & 20 \\
    User Convenience & 10 \\
    \textbf{Total} & \textbf{100} \\
    \bottomrule
\end{tabular}

\subsubsection*{Decision Table}
\resizebox{\textwidth}{!}{%
\begin{tabular}{lcccc}
    \toprule
    Criteria & Weight & Design 1 (Basic) & Design 2 (Mid-Range) & Design 3 (Advanced) \\
    \midrule
    Security & 40 & 6 (240) & 8 (320) & 10 (400) \\
    Cost & 30 & 10 (250) & 6 (150) & 3 (75) \\
    Power Consumption & 20 & 6 (90) & 8 (120) & 10 (150) \\
    User Convenience & 10 & 3 (30) & 6 (60) & 9 (90) \\
    \textbf{Total Score} & \textbf{100} & \textbf{680} & \textbf{720} & \textbf{780} \\
    \bottomrule
\end{tabular}%
}

\textbf{Best Design:} Design 3 (Advanced) with 780 points, prioritizing security, cost-efficiency, and power optimization.


\newpage
\section{Manufactured Product Testing}

\subsection*{Common Scenarios (Tests 1–10)}

\subsection*{1. Valid PIN Entry Test}
\subparagraph{Test Goals and Purpose}
\begin{itemize}
    \item Verify that a recognized 4-digit PIN unlocks the door without unnecessary delay.
    \item Check that the system logs this event as a successful access in Firestore.
\end{itemize}
\subparagraph{How We Test It}
\begin{itemize}
    \item Select a PIN known to be valid from the internal list.
    \item Enter it once and observe the system's response, both physical and digital.
    \item Look for updates to the LOCK\_STATE value in Firestore.
\end{itemize}
\subparagraph{Expectations of Test}
\begin{itemize}
    \item The bolt should retract within roughly half a second (this might vary slightly).
    \item LOCK\_STATE is set to 1 in Firestore upon unlock.
    \item There are no anomalies or unexpected entries in the logs.
\end{itemize}

\subsection*{2. Invalid PIN Lockout Test}
\subparagraph{Test Goals and Purpose}
\begin{itemize}
    \item Ensure the system prevents further PIN attempts after multiple consecutive failures.
    \item Confirm some form of lockout indication is presented, such as a light or alarm.
\end{itemize}
\subparagraph{How We Test It}
\begin{itemize}
    \item Enter three clearly incorrect PINs in under 30 seconds.
    \item Check the LOCKOUT\_COUNT value and observe for a visible or audible signal.
\end{itemize}
\subparagraph{Expectations of Test}
\begin{itemize}
    \item PIN entries are blocked for about 60 seconds following the third failure.
    \item Lockout state is visually or audibly indicated (could be LED or buzzer).
    \item Firestore logs each incorrect attempt, along with the lockout status.
\end{itemize}

\subsection*{3. Emergency PIN Test}
\subparagraph{Test Goals and Purpose}
\begin{itemize}
    \item Test whether a designated emergency PIN can override a lockout state.
    \item Ensure that using the emergency PIN resets the system's lockout logic.
\end{itemize}
\subparagraph{How We Test It}
\begin{itemize}
    \item Trigger a lockout by intentionally failing three PIN entries.
    \item Once lockout is active, input the emergency PIN.
    \item Monitor Firestore to verify an EMERGENCY\_UNLOCK entry is created.
\end{itemize}
\subparagraph{Expectations of Test}
\begin{itemize}
    \item The door should unlock immediately, without hesitation.
    \item Firestore logs the use of the emergency PIN distinctly.
    \item The system clears any active lockout status to resume normal operation.
\end{itemize}


